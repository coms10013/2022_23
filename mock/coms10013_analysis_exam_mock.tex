%ps1.tex
%notes for the course Probability and Statistics COMS10011 
%taught at the University of Bristol
%2019_20 Conor Houghton conor.houghton@bristol.ac.uk

%To the extent possible under law, the author has dedicated all copyright 
%and related and neighboring rights to these notes to the public domain 
%worldwide. These notes are distributed without any warranty. 

\documentclass[11pt,a4paper]{scrartcl}
\typearea{12}
\usepackage{graphicx}
%\usepackage{pstricks}
\usepackage{listings}
\usepackage{color}
\lstset{language=C}
\usepackage{fancyhdr}
\pagestyle{fancy}
\lhead{\texttt{github.com/coms10013/2022\_23} and  \texttt{coms10013.github.io}}
\lfoot{COMS10013 - ws1 - Conor}
\begin{document}

\section*{COMS10013 - exam questions - 25 points}

All sections are worth five points; outline solutions are just outlines, you should give more workings!

\subsection*{Part 1}
This question is about the definition and properties of differenciation. In one sentence describe in words what $df/dt$ means [1 points]. Give the formal definition of the limit [1 point] and of $df/dt$ [1 point] and use this to argue [2 points] that
    $$
    \frac{d(f+g)}{dt}=\frac{df}{dt}+\frac{dg}{dt}
    $$
  
  \subsubsection*{Outline Solution}
  So $df/dt$ is the rate of change of $f$, it quantifies how much $f$ changes as $t$ changes, at a single point in $t$. The limit
  $$\lim_{t\rightarrow a}f(t)=b$$
  if and only if for every $\epsilon$ there exists a $\delta$ such that $|t-a|\le \delta$ implies $|f(t)-b|\le \epsilon$; with this machinery we have
  $$\frac{df}{dt}=\lim_{h\rightarrow 0}\frac{f(t+h)-f(t)}{h}$$
  and the linearity of the limit means
  $$\frac{d(f+g)}{dt}=\lim_{h\rightarrow 0}\frac{f(t+h)+g(t+h)-f(t)-g(t)}{h}=\lim_{h\rightarrow 0}\frac{f(t+h)-f(t)}{h}=\lim_{h\rightarrow 0}\frac{g(t+h)-g(t)}{h}$$ and the second term clearly gives the answer.

  
  \subsection*{Part 2}
  This question is about actually taking derivatives. Using the quotient rule work out [1 point] 
  $$
    \frac{d\tan{x}}{dx}
    $$
    Using the chain rule work out [2 points]
    $$
    \frac{d\tan{\sin{x}}}{dx}
      $$
      Given the Taylor expansion of the exponential
      $$
      e^x=\sum_{n=0}^\infty \frac{x^n}{n!}
      $$
      show that [2 points]
      $$
      \frac{de^x}{dx}=e^x
      $$
      
    \subsubsection*{Outline Solution}
    So you use $\tan{x}=\sin{x}/\cos{x}$ so
    $$
        \frac{d\tan{x}}{dx}=\frac{\cos^2{x}+\sin^2{x}}{\cos^2{x}}=1/\cos^2{x}
    $$
        Next
 $$
    \frac{d\tan{\sin{x}}}{dx}=\frac{\cos{x}}{\cos^2{\sin{x}}}
      $$
    Finally,
    $$\frac{1}{n!}\frac{dx^n}{dx}=\frac{1}{(n-1)!}x^{n-1}$$
    so differentiating just shuffles all the terms down one.


      \subsection*{Part 3}
      This question is about partial derivatives. What is the gradient [1 point] of
      $$
      z=xy\cos{xy}
      $$
      Find the Hessian [2 point], show $(0,0)$ is an extremum [1 point] and indicate whether it is a maximum, minimum or saddle point [1 point].

      \subsubsection{Outline Solution}
      So the gradient is
      $$\nabla z=\left(y\cos{xy}-xy^2\sin{xy},x\cos{xy}-x^2y\sin{xy}\right)$$
      The Hessian is
      $$H=\left[\begin{array}{cc}z_{xx}&z_{xy}\\z_{yx}&z_{yy}\end{array}\right]$$
      where
      $$z_{xx} = -2y^2\sin{xy}-xy^3\cos{xy}$$
      and
      $$z_{xy} = \cos{xy}-3xy\sin{xy}-x^2y^2\cos{xy}=z_{yx}$$
      and finally
$$z_{yy}= -2x^2\sin(xy) - x^3y\cos(xy)$$
      So the gradient is zero at $x=y=0$ showing it is an extemum and
      $$H=\left[\begin{array}{cc}0&0\\0&0\end{array}\right]$$
      which has determinant $-1$ and trace zero, so the two eigenvalues have opposite signs and this is a saddle point.

      \subsection*{Part 4}
      This question is about complex numbers. What are the solutions of [1 point]
      $$z^2+2z+2=0$$
      What are the solutions of [1 point]
      $$z^2+2z+5=0$$
      In each case what is the relationship between the two solutions as complex numbers? [2 points]
      How many solutions do you expect for [1 point]
      $$z^{6}-1=0$$
      What are these solutions? [1 point]

      \subsubsection*{Outline Solution}
      So the first equation has solutions $z=-1\pm i$ and the second $z=-1\pm 2i$; in each case the solutions are conjugates of each other. $z^6=1$ has six solutions, they are
      \begin{equation}
        z=\exp{n\pi i/3}
      \end{equation}
      for $n$ between zero and five, inclusive.

\subsection*{Part 5}
This question is about differential equations. Solve the following [a-c 1 point each, d 2 points] where the dot means the derivative with respect to time.
	\begin{itemize}
		\item[(a)] $\dot{y}(t) - y(t) = 0$ with initial condition $y(0) = 1$.
		\item[(b)] $\dot{y}(t) + 3y(t) = 0$ with initial condition $y(3) = 3$.
		\item[(c)] $\dot{y}y(t) = 0$ with  initial condition $y(5) = 2$.
                  \item[(d)] $\dot{y}=ry(1-y)$ with $y(0)=y_0$. 
	\end{itemize}


\subsubsection*{Solution}
These previously appeared in worksheet 4 and they have solutions there, except d) for which we have
\begin{equation}
  \frac{1}{y(1-y)} \dot{y}=r
\end{equation}
and then using the partial fraction expansion
\begin{equation}
  \left(\frac{1}{y}+\frac{1}{1-y}\right)\dot{y}=r
\end{equation}
and so
\begin{equation}
  \int{\frac{dy}{y}}+\int{\frac{dy}{1-y}}=rt+C
\end{equation}
and hence
\begin{equation}
  \log{y}-\log{1-y}=rt+C
\end{equation}
or
\begin{equation}
  \log{\frac{y}{1-y}}=rt+c
\end{equation}
and hence
\begin{equation}
  \frac{y}{1-y}=Ae^{rt}
\end{equation}
and solve for $y$ to get
\begin{equation}
  y=\frac{Ae^{rt}}{1+Ae^{rt}}
  \end{equation}
\end{document}
