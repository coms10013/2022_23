%ps1.tex
%notes for the course Probability and Statistics COMS10011 
%taught at the University of Bristol
%2019_20 Conor Houghton conor.houghton@bristol.ac.uk

%To the extent possible under law, the author has dedicated all copyright 
%and related and neighboring rights to these notes to the public domain 
%worldwide. These notes are distributed without any warranty. 

\documentclass[11pt,a4paper]{scrartcl}
\typearea{12}
\usepackage{graphicx}
%\usepackage{pstricks}
\usepackage{listings}
\usepackage{color}
\lstset{language=C}
\usepackage{fancyhdr}
\pagestyle{fancy}
\lhead{\texttt{github.com/coms10013/2022\_23} and  \texttt{coms10013.github.io}}
\lfoot{COMS10013 - ws1 - Conor}
\begin{document}

\section*{COMS10013 - exam questions - 25 points}

All sections are worth five points; outline solutions are just outlines, you should give more workings!

\subsection*{Part 1}
This question is about the definition and properties of differenciation. In one sentence describe in words what $df/dt$ means [1 points]. Give the formal definition of the limit [1 point] and of $df/dt$ [1 point] and use this to argue [2 points] that
    $$
    \frac{d(f+g)}{dt}=\frac{df}{dt}+\frac{dg}{dt}
    $$

  
  \subsection*{Part 2}
  This question is about actually taking derivatives. Using the quotient rule work out [1 point] 
  $$
    \frac{d\tan{x}}{dx}
    $$
    Using the chain rule work out [2 points]
    $$
    \frac{d\tan{\sin{x}}}{dx}
      $$
      Given the Taylor expansion of the exponential
      $$
      e^x=\sum_{n=0}^\infty \frac{x^n}{n!}
      $$
      show that [2 points]
      $$
      \frac{de^x}{dx}=e^x
      $$

      \subsection*{Part 3}
      This question is about partial derivatives. What is the gradient [1 point] of
      $$
      z=xy\sin{xy}
      $$
      Find the Hessian [2 point], show $(0,0)$ is an extremum [1 point] and indicate whether it is a maximum, minimum or saddle point [1 point].

      \subsection*{Part 4}
      This question is about complex numbers. What are the solutions of [1 point]
      $$z^2+2z+2=0$$
      What are the solutions of [1 point]
      $$z^2+2z+5=0$$
      In each case what is the relationship between the two solutions as complex numbers? [2 points]
      How many solutions do you expect for [1 point]
      $$z^{6}-1=0$$
      What are these solutions? [1 point]

\subsection*{Part 5}
This question is about differential equations. Solve the following [a-c 1 point each, d 2 points] where the dot means the derivative with respect to time.
	\begin{itemize}
		\item[(a)] $\dot{y}(t) - y(t) = 0$ with initial condition $y(0) = 1$.
		\item[(b)] $\dot{y}(t) + 3y(t) = 0$ with initial condition $y(3) = 3$.
		\item[(c)] $\dot{y}y(t) = 0$ with  initial condition $y(5) = 2$.
                  \item[(d)] $\dot{y}=ry(1-y)$ with $y(0)=y_0$. 
	\end{itemize}


\end{document}
