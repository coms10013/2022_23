%ps1.tex
%notes for the course Probability and Statistics COMS10011 
%taught at the University of Bristol
%2019_20 Conor Houghton conor.houghton@bristol.ac.uk

%To the extent possible under law, the author has dedicated all copyright 
%and related and neighboring rights to these notes to the public domain 
%worldwide. These notes are distributed without any warranty. 

\documentclass[11pt,a4paper]{scrartcl}
\typearea{12}
\usepackage{graphicx}
%\usepackage{pstricks}
\usepackage{listings}
\usepackage{color}
\lstset{language=C}
\usepackage{fancyhdr}
\usepackage{amsmath}
\usepackage{amssymb}
\usepackage{amsthm}
\pagestyle{fancy}
\lhead{\texttt{github.com/coms10013/2022\_23} and  \texttt{coms10013.github.io}}
\lfoot{COMS10013 - ws6 - more solutions - Conor}
\begin{document}

\section*{COMS10013 - Analysis - WS6 - more outline solutions}
        
\subsection*{Extra questions}

\begin{enumerate}
\item \textbf{Divergence and Laplacian}. The divergence of a vector is
  a sort of complement to the gradient of a function. If
  \begin{equation}
    \mathbf{v}(x,y)=[v_1(x,y),v_2(x,y)]
  \end{equation}
  is a vector function in two dimensions then the divergence is
  \begin{equation}
    \textrm{div}\mathbf{v}=\frac{\partial v_1}{\partial x}+\frac{\partial v_2}{\partial y}
  \end{equation}
  This can be written as
  \begin{equation}
    \textrm{div}\mathbf{v}=\nabla\cdot\mathbf{v}
  \end{equation}
  Now if you have a function $f(x,y)$ we define the Laplacian as
  \begin{equation}
    \Delta f=\nabla\cdot\nabla(f)=\textrm{div}\,\textrm{grad}\,f
  \end{equation}
  write this down in terms of partial derivatives. How is it related to the Hessian? So this is easy enough:
  \begin{equation}
    \Delta f=\nabla\cdot [f_x,f_y]=f_{xx}+f_{yy}
  \end{equation}
  and this is the trace of the Hessian.

  
 
 \item \textbf{Curl}. After gradient and divergence, the curl fills out the list of vector differential operators, weirdly it only exists in three dimensions. If 
\begin{equation}
    \mathbf{v}=[v_1,v_2,v_3]
  \end{equation}
is a vector function then
\begin{equation}
    \textrm{curl}\,\mathbf{v}=\left[\frac{\partial v_3}{\partial y}-\frac{\partial v_2}{\partial z},\frac{\partial v_1}{\partial z}-\frac{\partial v_3}{\partial x},\frac{\partial v_2}{\partial x}-\frac{\partial v_1}{\partial y}\right]
  \end{equation}
if $f(x,y,z)$ is a function and $\textbf{v}(x,y,z)$ a vector function, both in three dimensions what are $\textrm{curl}\,\textrm{grad}\,f$ and $\textrm{div}\,\textrm{curl}\,\textbf{v}$? Again, this is easy that it sounds, I think.
\begin{equation}
  \textrm{curl}\,\textrm{grad}\,f=\textrm{curl}\,[f_x,f_y,f_z]
\end{equation}
and hence
\begin{equation}
  \textrm{curl}\,\textrm{grad}\,f=[f_{zy}-f_{yz},f_{xz}-f_{zx},f_{yx}-f_{xy}]=\mathrm{0}
\end{equation}
For the other one
\begin{equation}
  \textrm{div}\,\textrm{curl}\,\textbf{v}=\textbf{div}(v_{3y}-v_{2z},v_{1z}-v_{3x},v_{2x}-v_{1y})=v_{3yx}-v_{2zx}+v_{1zy}-v_{3xy}+v_{2xz}-v_{1yz}
\end{equation}
and, since the order of the differentiation doesn't matter, this all cancels. These formula play a surprisingly important role in differential topology, which studies the relationships between differential operators, like these, and the shapes of spaces.

\item \textbf{Quaternions}. The quaternions are a type of generalization of complex numbers; they have some deep mathematical properties but in practice they are usually used to help describe rotations in three-dimensional space. Instead of just $i$ there are three imaginary numbers $i$, $j$ and $k$, and these all square to minus one: $i^2=j^2=k^2=-1$. In addition $ijk=-1$ and the numbers are \textsl{anti-commutative}: $ij=-ji$, $jk=-kj$ and so on. Lots of other relationships can be derived from these rules, for example if you multiply $ijk=-1$ you get $jk=i$, or if you switch it $jik=1$ and multiply by $j$ you get $ik=1$. Anyway, if
  \begin{equation}
    q=w+xi+yj+zk
  \end{equation}
  and
  \begin{equation}
    q^*=w-xi-yj-zk
  \end{equation}
  prove
  \begin{equation}
    qq^*=w^2+x^2+y^2+z^2
  \end{equation}
 This is just a big calculation and everything cancels! 
        
\end{enumerate}



\end{document}

