%ps1.tex
%notes for the course Probability and Statistics COMS10011 
%taught at the University of Bristol
%2019_20 Conor Houghton conor.houghton@bristol.ac.uk

%To the extent possible under law, the author has dedicated all copyright 
%and related and neighboring rights to these notes to the public domain 
%worldwide. These notes are distributed without any warranty. 

\documentclass[11pt,a4paper]{scrartcl}
\typearea{12}
\usepackage{graphicx}
%\usepackage{pstricks}
\usepackage{listings}
\usepackage{amsmath}
\usepackage{color}
\lstset{language=C}
\usepackage{fancyhdr}
\pagestyle{fancy}
\lhead{\texttt{github.com/coms10013/2022\_23} and  \texttt{coms10013.github.io}}
\lfoot{COMS10013 - ws1 - moreSolutions - Conor}
\begin{document}

\section*{COMS10013 - Analysis - WS1 - moreSolutions}

\subsection*{Extra questions}

These are extra questions you might attempt in the workshop or at a later time.

\begin{enumerate}

\item Differentiate $x^x$ with respect to $x$. \textbf{Solution}: so this is hard unless you know or spot the trick, which is to know that $\log a^b=b\log a$, and that $\exp{\log{a}}=a$. In this case we do
  \begin{equation}
    x^x=\exp{\log{x^x}}=\exp{\left(x\log{x}\right)}
  \end{equation}
  and now we have something we can differentiate, in this case using the chain rule and
  \begin{equation}
    \frac{d}{dx}x\log{x}=\log{x}+1
  \end{equation}
  This means
  \begin{equation}
    \frac{dx^x}{dx}=(1+\log{x})x^x
  \end{equation}
  

\item The function $z(x, y) = x^2 + y^2 + 2x - 3y$ has a global minimum. Find this by taking
  the gradient and searching for the point where the gradient is zero. \textbf{Solution}: well the gradient is
  \begin{equation}
    \text{grad}(z)=(2x+2,2y-3)
  \end{equation}
  so that's equal to $(0,0)$ when $x=-1$ and $y=3/2$. 
\item Check that this point you found really is a minimum by computing the Hessian of the
  function at this point, and checking that it is positive definite, that is, all eigenvalues are positive. \textbf{Solution}: so the Hessian is
  \begin{equation}
    H=\left(\begin{array}{cc}2&0\\0&2\end{array}\right)
  \end{equation}
  so, trivially, the eigenvalues are both two and this is a positive definite Hessian.

\end{enumerate}

\end{document}
