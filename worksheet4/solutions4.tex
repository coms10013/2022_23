%ps1.tex
%notes for the course Probability and Statistics COMS10011 
%taught at the University of Bristol
%2019_20 Conor Houghton conor.houghton@bristol.ac.uk

%To the extent possible under law, the author has dedicated all copyright 
%and related and neighboring rights to these notes to the public domain 
%worldwide. These notes are distributed without any warranty. 

\documentclass[11pt,a4paper]{scrartcl}
\typearea{12}
\usepackage{graphicx}
%\usepackage{pstricks}
\usepackage{listings}
\usepackage{color}
\lstset{language=C}
\usepackage{fancyhdr}
\usepackage{amsmath}
\usepackage{amssymb}
\usepackage{amsthm}
\pagestyle{fancy}
\lhead{\texttt{github.com/coms10013/2022\_23} and  \texttt{coms10013.github.io}}
\lfoot{COMS10013 - ws4 - Conor}
\begin{document}

\section*{COMS10013 - Analysis - WS4 - outline solutions}

This is for the main questions, solutions for the other questions will come later.
  
\subsection*{Questions}

These are the questions you should make sure you work on in the workshop.

\begin{enumerate}

\item \textbf{A linear accelerated motion question}. A train is travelling from Bristol to London Paddinton at the maximum speed of 55.9 m/s, 125 mph, when the driver activates the emergency break. This causes the train to decelerate uniformly at 1.2 m/s$s^2$. How far will the train travel until it stops and how long will this take, in seconds. Do this using differential equations, for example:
  \begin{equation}
    \frac{dv}{dt}=-1.2
  \end{equation}
  not by looking up formulas.
	\\
	\textbf{Solution:}
	\\
	Let 
	\begin{itemize}
		\item $y(t)$ be the position of the train at time $t$,
		\item $v(t) = \frac{dy}{dt}$ be the velocity of the train at time $t$.
		\item $a(t) = \frac{dv}{dt} = \frac{d^2y}{dt^2}$ be the acceleration of the train at time $t$.
	\end{itemize}
	It is given that $y(0) = 0$, $v(0) = 55.9$, and that for $t\geq 0$ the acceleration $a(t) = -1.2$ is constant.
	Then 
	$$v(t) = \int a(t) dt = \int -1.2 \, dt = -1.2t + v(0) = -1.2t + 55.9.$$
	So the train reaches velocity $v = 0$ at time $t = 55.9/1.2 \approx 46.6$,
	that is, after 46.6 seconds.
	Also the position
	$$y(t) = \int v(t) dt = \int -1.2t + 55.9 \, dt = 
	-0.6 t^2 + 55.9 t + y(0) = -0.6 t^2 + 55.9 t.$$
	So the position of $y$ at the moment the velocity $v$ reaches zero,
	which we just computed occurs at time $t = 55.9/1.2$, is
	$$y(55.9/1.2) = -0.6 \cdot (55.9/1.2)^2 + 55.9 \cdot (55.9/1.2) \approx 1302.0,$$
	i.e., 1302 metres further on from the point when the brake was activated.


  
\item \textbf{Types of differential equations}
  Write down (but don't solve) an example of a differential equation that is:
  \begin{itemize}
  \item[(a)] First-order, linear but not homogeneous, with constant coefficients.
  \item[(b)] First-order, linear, homogeneous but without constant coefficients.
  \item[(c)] Second-order, linear, homogeneous, with constant coefficients.
  \item[(d)] Second-order, linear, not homogeneous, without constant coefficients.
  \end{itemize}
	\textbf{Solution:}
	\begin{itemize}
		\item[(a)] Anything of the form
		\[ay'(x) + by(x) = c,\]
		where $a,b,c$ are independent of $x$ and $y$, and $a$ and $c$ are non-zero.
		\item[(b)] Anything of the form
		\[a(x)y'(x) + b(x)y(x) = 0,\]
		where at least one of $a(x)$ and $b(x)$ is non-constant, and $a$ is non-zero.
		\item[(c)] Anything of the form
		\[ay''(x) + by'(x) + cy(x) = 0,\]
		where $a,b,c$ are independent of $x$ and $y$ and $a$ is non-zero.
		\item[(d)] Anything of the form
		\[a(x)y''(x) + b(x)y'(x) + c(x)y(x) = d(x),\]
		where at least one of $a(x), b(x), c(x), d(x)$ is non-constant, and
		$a(x)$ and $d(x)$ are non-zero.
	\end{itemize}


  
\item \textbf{Differential equations} Solve the following, linear,
  homogeneous, first-order, constant coefficients, differential
  equations once using separation of variables and once with the
  \emph{ansatz}.
	\begin{itemize}
		\item[(a)] $\dot{y}(t) - y(t) = 0$ with initial condition $y(0) = 2$: this is the classic: $y=A\exp{t}$ and the initial condition gives $A=2$.
		\item[(b)] $\dot{y}(t) + 3y(t) = 0$ with initial condition $y(3) = 3$: this one isn't much different, $y=A\exp{-3t}$ and the intial conditionisn't an initial condition, which is sneaky, but $A\exp{-9}=3$ so $A=3\exp{9}$.
		\item[(c)] $\dot{y}y(t) = 0$ with  initial condition $y(5) = 2$: oh dear, this was typo, but lets pretend it wasn't.
                  $$\dot{y}y=\frac{1}{2}\frac{dy^2}{dt}$$
                  so integrating give $y^2=C$ or, with some sloppiness renaming the arbitrary constant, $y=C$ and then $y(5)=2$ tells us that $C=2$. 
		\item[(d)] $\dot{y}(t) + 5y(t) = 0$ with initial condition
                  $y(1) = 1$. Not sure what the point here is, it's the same as the others really, $y=A\exp{(-5t)}$ and the condition tells us that $A=\exp{5}$.
	\end{itemize}

\end{enumerate}

\end{document}

