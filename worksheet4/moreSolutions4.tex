%ps1.tex
%notes for the course Probability and Statistics COMS10011 
%taught at the University of Bristol
%2019_20 Conor Houghton conor.houghton@bristol.ac.uk

%To the extent possible under law, the author has dedicated all copyright 
%and related and neighboring rights to these notes to the public domain 
%worldwide. These notes are distributed without any warranty. 

\documentclass[11pt,a4paper]{scrartcl}
\typearea{12}
\usepackage{graphicx}
%\usepackage{pstricks}
\usepackage{listings}
\usepackage{color}
\lstset{language=C}
\usepackage{fancyhdr}
\usepackage{amsmath}
\usepackage{amssymb}
\usepackage{amsthm}
\pagestyle{fancy}
\lhead{\texttt{github.com/coms10013/2022\_23} and  \texttt{coms10013.github.io}}
\lfoot{COMS10013 - ws4 - more solutions - Conor}
\begin{document}

\section*{COMS10013 - Analysis - WS4 - more outline solutions}

\subsection*{Extra questions}

These are extra questions you might attempt in the workshop or at a
later time; in fact these questions are tricky so you might want to
come back to them later when you've had some more lectures.

\begin{enumerate}

	
	\item \textbf{Solutions as a vector space}	
	The aim of this exercise is to prove most of the following theorem:
	the solutions to the second-order linear homogeneous differential equation
	$a\ddot{y}(t) + b\dot{y}(t) + cy(t) = 0$ form a vector space. Note that this also follows from a theorem stated in the lecture notes. Prove:
	\begin{itemize}
		\item[(a)] If $f(t)$ and $g(t)$ are two solutions to this differential equation, then $h(t) = f(t) + g(t)$ is also a solution to this differential equation: this is easy since $d(f+g)/dt=\dot{f}+\dot{g}$ and so on.
		\item[(b)] If $f(t)$ is a solution to this differential equation,
		and $s$ is any integer, then $k(t) = s \cdot f(t)$ is also a solution to this differential equation: again, follow from linearity of differentiation $d(sf)/dt=s\dot{f}$.
		\item[(c)] The function $f_0(t) = 0$ (the function that is zero for all $t$) is a solution to the differential equation: easy to see from substituting zero into the equation.
	\end{itemize}
	Technically we would also have to show that addition of
        solutions is commutative and associative, but this is tedious
        and doesn't have anything to do with differential equations -
        the way to prove this would just be to show it for all
        functions. So you don't have to prove that here.
	
	Note that this theorem holds even if $a,b,c$ are functions of
        $t$, same proof, you never had to differentiate these
        constants, and for any order of linear differential equation,
        not just second order. The theorem doesn't hold for
        inhomogeneous equations however: for a challenge, can
        you see which parts of the proof don't work for $a\ddot{y}(t) +
        b\dot{y}(t) + cy(t) = d$ and for which $d$? \textbf{Solution}: so if $f$ and $g$ are solutions and you substitute in $\lambda f+\mu g$ you end up with $(\lambda +\mu)d=d$ so, basically, the linearity holds for $\mu=1-\lambda$; the third property, zero is a solution, doesn't hold unless $d$ is zero. 
	
	\item \textbf{Radioactive decay}
	This is the standard example from physics to motivate differential equations.
	Marie Sk\l{}odowska–Curie discovered the element Radium in the late 19$^\text{th}$ century, together with her husband Pierre. Her notebooks on which she recorded her discoveries are so radioactive that they are kept locked in lead boxes.
	
	An atom of Radium-226 normally decays into an alpha particle,
        a Helium nucleus, and an atom of Radon-222. Each atom has a
        fixed probability of decaying in a fixed time period, so if
        you have a box of radium then the number of decays you will
        observe in a fixed, small time period is proportional to the
        number of atoms of radium you had to start with.
	
	The standard way to write this is $\frac{dy}{dt} = cy$, where
        $c$ is a negative constant, or more suggestively $dy = cy\,dt$
        which exactly captures the following statement: the change in
        the number of atoms ($dy$) that you observe is proportional
        (via $c$) to the number of atoms you started with ($y$) and
        the change in time ($dt$) during your observation, e.g., the
        length of time you observe for.  We recognise this as a
        differential equation and can immediately derive the
        radioactive decay equation $y(t) = Ae^{-rt}$ where $A = y(0)$
        is the number of atoms you started with and $r = -c$ is the
        rate of decay ($r$ is positive).
	
	In nuclear physics, the half-life $\lambda$ is the time it
        takes for half of a given quantity of radioactive substance to
        decay.
	
	\begin{itemize}
		\item[(a)] Assuming $t$ is measured in years, find a
                  formula for the half-life $\lambda$ in years as a
                  function of the decay constant $r$ and vice versa.
		\item[(b)] Given that Radon-226 has a half-life of
                  1600 years, what is its rate of decay $r$?
	\end{itemize}		
        \textbf{Solution}: so if $y(t)=Ae^{-rt}$ putting $t=0$ tells us that $y(t)=y(0)e^{-rt}$, or, equivalently,
        \begin{equation}
          \log{\frac{y(t)}{y(0)}}=-rt
        \end{equation}
        where we have used $\log{a}-\log{b}=\log{a/b}$. Now this means
        \begin{equation}
          \log{2}=r\lambda
        \end{equation}
        or $\lambda=\log{2}/\lambda$; if $\lambda=1600$ this means $r=0.00043$ per year.

        
        \item \textbf{A non-linear example} The example from physics
          about is about decay, the obvious corresponding example from
          biology population growth, if every squirrel has $r$ baby
          squirrels each year and squirrels can start having babies as
          soon as they are born and leaving out squirrel death then
          the number of squirrels satisfies:
          \begin{equation}
            \frac{dN}{dt}=rN
          \end{equation}
          and since the solution is $N=A\exp{rt}$ this explodes in a
          Malthusian disaster with a one metre deep layer of squirrels coating the earth after only
          \begin{equation}
            t=(\log{96000000}-\log{A})/r=18/r
          \end{equation}
          years, which even for $r=1$ is a problem; a child born today
          would drown in squirrels before reaching adulthood.\\ \\ Now
          there are a number of incorrect approximations, the
          reproductive precociousness and immortality of squirrels
          but, while these will change the precise predicted time to
          the squirrel singularity, they won't stop it. However, we
          realise that the differential equation ignores the resource
          requirements of squirrels, their population is limited by
          the availability of nuts; if there are too many squirrels
          there aren't enough nuts for them all and the population
          growth slows. In the language of ecology there is a limited
          \textsl{carrying capacity} for squirrels.\\
          \\
          This is reflected in the logistic equation introduced by Pierre Fran\c{c}ois Verhulst in 1838; this equation looks like
          \begin{equation}
            \frac{df}{dt}=rf(1-f)
          \end{equation}
In this version the carrying capacity has been set to one so $f$ is
the population as a fraction of the carrying capacity; the equation
says that the rate of growth of $f$ depends on $f$, the number of
squirrels, and on $1-f$, the amount of resource not already consumed,
the amount of uneaten nuts. Notice that when $f$ gets close to one the
growth of the population slows.\\
\\
This equation is non-linear so it can't be solved by the methods used for linear equations. Many important non-linear equations can't be solved but this one case, by direct integration. The fact you need is that
\begin{equation}
  \frac{1}{f(1-f)}=\frac{1}{f}+\frac{1}{1-f}
\end{equation}
\textbf{Solution}:
 we have
\begin{equation}
  \frac{1}{y(1-y)} \dot{y}=r
\end{equation}
and then using the partial fraction expansion
\begin{equation}
  \left(\frac{1}{y}+\frac{1}{1-y}\right)\dot{y}=r
\end{equation}
and so
\begin{equation}
  \int{\frac{dy}{y}}+\int{\frac{dy}{1-y}}=rt+C
\end{equation}
and hence
\begin{equation}
  \log{y}-\log{1-y}=rt+C
\end{equation}
or
\begin{equation}
  \log{\frac{y}{1-y}}=rt+c
\end{equation}
and hence
\begin{equation}
  \frac{y}{1-y}=Ae^{rt}
\end{equation}
and solve for $y$ to get
\begin{equation}
  y=\frac{Ae^{rt}}{1+Ae^{rt}}
  \end{equation}



          
        
\end{enumerate}



\end{document}
     
