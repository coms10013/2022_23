\documentclass[12pt]{article}
\usepackage{amsfonts, epsfig}
\usepackage[authoryear]{natbib}
\usepackage{graphicx}
\usepackage{fancyhdr}
\usepackage{amsmath}
\pagestyle{fancy}
\lfoot{\texttt{coms10013.github.io}}
\lhead{Analysis - 2 Partial derivatives - Conor}
\rhead{\thepage}
\cfoot{}
\begin{document}

\section*{Many dimensions}

In the previous lecture we looked at differenciation, but only in one
dimension, $f(x)$ for example. Often we are interested in function of
more than dimension, $z(x,y)$, the height of the ground at a point
$(x,y)$; the functions of physics, temperature for example, that have
a value for every point $(x,y,z)$, the loss function of an deep
learning network that depends on million of parameters. All of these
and many more. It is natural to ask how differentiation works for
functions of more than one variable, the laws of physics depends on
it, as does optmization a deep learning network.

In fact, the definitions start out fairly straight forward. The derivative with respect to $x$ tells us the rate of change as $x$ changes, say $f(x,y)$:
\begin{equation}
  \frac{\partial f}{\partial x}=\lim_{h\rightarrow 0}\frac{f(x+h,y)-f(x)}{h}
\end{equation}
Basically the $y$ does nothing while the derivative in $x$ is being
calculated. You will notice that the notation has changed slightly,
with $\partial f/\partial x$ having the curly $d$, pronounced ``del''
or ``partial'' and we call this the \textbf{partial derivative}.

This is an old notation intended to distinguish what we are looking at
here, partial derivatives, from the so called total derivative. This
happens when one of the variables depends on the other, say, for
example you have a path in $(x,y)$ space you might write it as
$(x,y(x))$ so changing $x$ will change the function in two ways, one
way is because $x$ itself changes and a second because changing $x$
changes $y$; in this case the derivative is called the \textbf{total
  derivative}; in fact finding the total derivative is not that common
and we won't consider it here, but the notational difference between
differentiating in one dimension and in many is annoying since it is
essentially the same calculation and because there are times when you
have a function with variables and parameters and you don't know which
notation to use. Anyway, try not to worry about the notation, as is
typical in mathematics it is powerful and useful but not as clear cut
as you'd expect and at times annoying.

Anyway, obviously we can also define the partial derivative with respect to $y$:
\begin{equation}
  \frac{\partial f}{\partial y}=\lim_{h\rightarrow 0}\frac{f(x,y+h)-f(x)}{h}
\end{equation}
It is sometimes useful to write a shorthand
\begin{equation}
  f_x=\frac{\partial f}{\partial x}
\end{equation}
and
\begin{equation}
  f_y=\frac{\partial f}{\partial y}
\end{equation}
As with the dots and primes in one-dimensional, the proper
fraction-like notation is clearly better since it expresses in
notation some of the properties of the derivative. However, the other
notation is also commonly used because we are lazy.

Here is an example:
\begin{equation}
  f(x,y)=\sin{x}\cos{y}
\end{equation}
and $f_x=\cos{x}\cos{y}$ whereas $f_y=-\sin{x}\sin{y}$.





\section*{Summary}

This set of notes revises basic calculus using the old-fashioned
notion of infinitessimals. We gave a list of standard derivatives and
looked at the chain rule. 

\end{document}

